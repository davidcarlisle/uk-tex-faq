% $Id: faq-biblio.tex,v 1.15 2014/01/28 18:17:36 rf10 Exp rf10 $

\section{Bibliographies and citations}

\subsection{Creating bibliographies}

\Question[Q-buildbib]{Creating a \BibTeX{} bibliography file}

A \BibTeX{} bibliography file may reasonably be compared to a small
database, the entries in which are references to literature that may
be called up by citations in a document.

Each entry in the bibliography has a \emph{type} and a unique
\emph{key}.  The bibliography is read, by \BibTeX{}, using the details
specified in a \emph{bibliography style}.  From the style, \BibTeX{}
finds what entry types are permissible, what \emph{fields} each entry
type has, and how to format the whole entry.

The type specifies the type of document you're making reference to; it
may run all the way from things like ``\environment{Book}'' and
``\environment{Proceedings}'' (which may even contain other citations
of type ``\environment{InBook}'' or  ``\environment{InProceedings}'')
through dissertation styles like  ``\environment{PhdThesis}'' to
otherwise-uncategorisable things such as ``\environment{Misc}''.  The
unique key is something you choose yourself: it's what you use when
you want to \Qref*{cite an entry in the file}{Q-usebibtex}.  People
commonly create a key that combines the (primary) author's name and
the year of publication, possibly with a marker to distinguish
publications in the same year.  So, for example, the Dyson, Eddington,
Davidson paper about deflection of starlight appears in my
experimental \extension{bib} file as \texttt{Dyson20.1}.

So, noting the rules of the style, you have `simply' to write a
bibliography database.  Fortunately, there are several tools to help
in this endeavour:
\begin{itemize}
\item Most of the better \Qref*{\AllTeX{}-oriented editors}{Q-editors}
  have ``\BibTeX{} modes''.
\item If you have an existing \environment{thebibliography}
  environment, the \ProgName{Perl} script \ProgName{tex2bib} will
  probably help.
\item There are a number of \BibTeX{} bibliography management systems
  available, some of which permit a graphical user interface to the
  task.  Sadly, none seems to be available with the ordinary \TeX{}
  distributions.
  
  Tools such as \ProgName{Xbibfile} (a graphical user interface),
  \ProgName{ebib} (a database application written to run `inside'
  \ProgName{emacs}) and 
  \ProgName{btOOL} (a set of \ProgName{perl} tools for building 
  \BibTeX{} database handlers) are available from \acro{CTAN}.

  Other systems, such as
  \href{http://refdb.sourceforge.net/}{\ProgName{RefDB}},
  \href{http://www.nongnu.org/biborb}{BibORB},
  \href{http://bibdesk.sourceforge.net/}{\ProgName{BibDesk}},
  \href{http://pybliographer.org}{\ProgName{pybliographer}} and the
  \ProgName{Java}-based
  \href{http://freshmeat.net/projects/bibkeeper/}{\ProgName{Bibkeeper}}
  and \href{http://jabref.sourceforge.net}{\ProgName{JabRef}} (which
  claims to supersede \ProgName{Bibkeeper})
  are only available from their development sites.
% review of web database offerings in textex_368
\item Some commercial citation-management systems will export in
  \BibTeX{} format; an example is
  \href{http://www.endnote.com/}{EndNote}.
\item Data from on-line citation databases may often be translated to
  \BibTeX{} format by utilities to be found on \acro{CTAN}.  For
  example, the \ProgName{Perl} script \ProgName{isi2bibtex} will
  translate citations from \acro{ISI} ``Web of knowledge'' (a
  subscription service, available to \acro{UK} academics via
  \acro{BIDS}).  UK academics may translate \acro{BIDS} downloads
  using \ProgName{bids.to.bibtex}
\item \href{http://scholar.google.com}{Google Scholar} provides an
  ``Import into \BibTeX{}'' tab for each reference it finds for you:
  that tab gives you a page containing a \BibTeX{} entry for the
  reference.
\end{itemize}
\begin{ctanrefs}
\item[bids.to.bibtex]\CTANref{bidstobibtex}
\item[btOOL]\CTANref{btOOL}
\item[ebib]\CTANref{ebib}
\item[isi2bibtex]\CTANref{isi2bibtex}
\item[tex2bib]\CTANref{tex2bib}
\item[tex2bib.readme]\CTANref{tex2bib-doc}
\item[xbibfile]\CTANref{xbibfile}
\end{ctanrefs}

\Question[Q-custbib]{Creating a bibliography style}

It \emph{is} possible to write your own: the standard bibliography
styles are distributed in a form with many comments, and there is a description
of the language in the \bibtex{} distribution (see % ! line wrap
\Qref[question]{BibTeX documentation}{Q-BibTeXing}).
However, it must be admitted that the language in which \BibTeX{}
styles are written is pretty obscure, and one would not recommend
anyone who's not a confident programmer to write their own, though
minor changes to an existing style may be within the grasp of many.

If your style isn't too `far out', you can probably avoid programming
it by using the facilities of the \Package{custom-bib} bundle.  The bundle
contains a file \File{makebst.tex}, which runs you through a text menu
to produce a file of instructions, which you can then use to generate your
own \extension{bst} file.  This technique doesn't offer entirely new styles
of document, but the \Package{custom-bib}'s ``master \BibTeX{}
styles'' already offer significantly more than the \BibTeX{} standard set.

An alternative, which is increasingly often recommended, to use
\Qref*{\Package{biblatex}}{Q-biblatex}.  \Package{Biblatex} offers
many hooks for adjusting the format of the output of your `basic'
\bibtex{} style, and a collection of `contributed' styles have also
started to appear.  Note.bowever There are not as many of
\Package{biblatex}'s contributed styles as there are for \bibtex{},
and there is no \Package{custom-biblatex}, both of which suggest that
beginners' r\"ole models are hard to come by.  As a result, beginners
should probably resist the temptation to write their own contributed
\package{biblatex} style.
\begin{ctanrefs}
\item[biblatex.sty]\CTANref{biblatex}
\item[biblatex \nothtml{\normalfont}contributed styles]\CTANref{biblatex-contrib}
\item[\nothtml{\normalfont}\BibTeX{} documentation]\CTANref{bibtex}
\item[makebst.tex]Distributed with \CTANref{custom-bib}
\end{ctanrefs}

\Question[Q-capbibtex]{Capitalisation in \BibTeX{}}

The standard \BibTeX{} bibliography styles impose fixed ideas about
the capitalisation of titles of things in the bibliography.  While
this is not unreasonable by \BibTeX{}'s lights (the rules come from
the \emph{Chicago Manual of Style}) it can be troublesome, since
\BibTeX{} fails to recognise special uses (such as acronyms, chemical
formulae, etc.).

The solution is to enclose the letter or letters, whose capitalisation
\BibTeX{} should not touch, in braces, as:
\begin{quote}
\begin{verbatim}
title = {The {THE} operating system},
\end{verbatim}
\end{quote}
Sometimes you find \BibTeX{} changing the case of a single letter
inappropriately.  No matter: the technique can be applied to single
letters, as in:
\begin{quote}
\begin{verbatim}
title = {Te{X}niques and tips},
\end{verbatim}
\end{quote}
If your document design specification requires a different style of
capitalisation, you should acquire a bibliography style that doesn't
enforce \BibTeX{}'s default rules.  It is definitely \emph{not} a good
idea to enclose an entire title in braces, as in
\begin{quote}
\begin{verbatim}
title = {{TeXniques and tips}},
\end{verbatim}
\end{quote}
though that does ensure that the capitalisation is not changed.  Your
\BibTeX{} database should be a general-purpose thing, not something
tuned to the requirements of a particular document or bibliography
style, or to the way you are thinking today~--- for example, on a
future occasion, you might find yourself using a different \BibTeX{}
style with different capitalisation rules.

There's more on the subject in the
% beware line wrap
\Qref*{\BibTeX{} documentation}{Q-BibTeXing}.

\Question[Q-bibaccent]{Accents in bibliographies}

\BibTeX{} not only has a tendency (by default) to mess about with the
% line wrap
\Qref*{case of letters in your bibliography}{Q-capbibtex},
also makes a hash of accent commands:
``\texttt{ma}\csx{\textasciitilde}\texttt{nana}'' comes out as ``ma
nana'' (!).  The solution is similar that of the letter case problem:
enclose the troublesome sequence in braces, as
``\texttt{\{}\csx{\textasciitilde}\texttt{n\}}'', in this example.

\Question[Q-bibstrtl]{`String too long' in \BibTeX{}}

The \BibTeX{} diagnostic ``Warning--you've exceeded 1000, the
\texttt{global-string-size}, for entry \texttt{foo}'' usually arises
from a very large abstract or annotation included in the database.
The diagnostic usually arises because of an infelicity in the coding of
\File{abstract.bst}, or styles derived from it.  (One doesn't
ordinarily output annotations in other styles.)

The solution is to make a copy of the style file (or get a clean copy
from \acro{CTAN}~--- \CTANref{abstract-bst}), and rename it (e.g., on a
long file-name system, to \File{abstract-long.bst}).  Now edit it: find
function \texttt{output.nonnull} and
\begin{itemize}
\item change its first line (line 60 in the version on \acro{CTAN})
  from
\begin{quote}
\begin{verbatim}
{ 's :=
\end{verbatim}
\end{quote}
to
\begin{quote}
\begin{verbatim}
{ swap$
\end{verbatim}
\end{quote}
Finally,
\item delete the function's last line, which just says ``\texttt{s}
  (line 84 in the version on \acro{CTAN}).
\end{itemize}
Finally, change your \csx{bibliographystyle} command to refer to the
name of the new file.

This technique applies equally to any bibliography style: the same
change can be made to any similar \texttt{output.nonnull} function.

If you're reluctant to make this sort of change, the only way forward
is to take the entry out of the database, so that you don't encounter
\BibTeX{}'s limit, but you may need to retain the entry because it
will be included in the typeset document.  In such cases, put the body
of the entry in a separate file:
\begin{quote}
\begin{verbatim}
@article{long.boring,
  author =    "Fred Verbose",
  ...
  abstract =  "{\input{abstracts/long.tex}}"
}
\end{verbatim}
\end{quote}
In this way, you arrange that all \BibTeX{} has to deal with is the
file name, though it will tell \TeX{} (when appropriate) to include
all the long text.

\Question[Q-manyauthor]{\BibTeX{} doesn't understand lists of names}

\BibTeX{} has a strict syntax for lists of authors' (or editors')
names in the \BibTeX{} data file; if you write the list of names in a
``natural''-seeming way, the chances are you will confuse \BibTeX{},
and the output produced will be quite different from what you had
hoped.

Names should be expressed in one of the forms
\begin{quote}
\begin{verbatim}
First Last
Last, First
Last, Suffix, First
\end{verbatim}
\end{quote}
and lists of names should be separated with ``\texttt{and}''.
For example:
\begin{quote}
\begin{verbatim}
AUTHOR = {Fred Q. Bloggs, John P. Doe \&
          Another Idiot}
\end{verbatim}
\end{quote}
falls foul of two of the above rules: a syntactically significant
comma appears in an incorrect place, and `\csx{\&}' is being used as a
name separator.  The output of the above might be something like:
\begin{quote}
\begin{wideversion}
\begin{verbatim}
John P. Doe \& Another Idiot Fred Q. Bloggs
\end{verbatim}
\end{wideversion}
\begin{narrowversion}
\begin{verbatim}
John P. Doe \& Another Idiot
                        Fred Q. Bloggs
\end{verbatim}
\end{narrowversion}
\end{quote}
because ``John P. Doe \& Another Idiot has become the `first name',
while ``Fred Q. Bloggs'' has become the `last name' of a single
person.  The example should have been written:
\begin{quote}
\begin{narrowversion}
\begin{verbatim}
AUTHOR = {Fred Q. Bloggs and
          John P. Doe and
          Another Idiot}
\end{verbatim}
\end{narrowversion}
\begin{wideversion}
\begin{verbatim}
AUTHOR = {Fred Q. Bloggs and John P. Doe and
          Another Idiot}
\end{verbatim}
\end{wideversion}
\end{quote}
Some bibliography styles implement clever acrobatics with very long
author lists.  You can force truncation by using the pseudo-name
``\texttt{others}'', which will usually translate to something like
``\emph{et al}'' in the typeset output.  So, if Mr.~Bloggs wanted to
distract attention from his co-authors, he would write:
\begin{quote}
\begin{verbatim}
AUTHOR = {Fred Q. Bloggs and others}
\end{verbatim}
\end{quote}

\Question[Q-citeURL]{\acro{URL}s in \BibTeX{} bibliographies}

There is no citation type for \acro{URL}s, \emph{per se}, in the
standard \BibTeX{} styles, though Oren Patashnik (the author of
\BibTeX{}) is believed to be considering developing one such for use
with the long-awaited \BibTeX{} version~1.0.

The actual information that need be available in a citation of an
\acro{URL} is discussed at some length in the publicly available
on-line 
\begin{narrowversion}
  extracts of \acro{ISO}~690--2, available via
  \URL{http://www.nlc-bnc.ca/iso/tc46sc9/standard/690-2e.htm};
  % checked 2004-01-15
\end{narrowversion}
\begin{wideversion}
\href{http://www.nlc-bnc.ca/iso/tc46sc9/standard/690-2e.htm}{extracts of \acro{ISO}\nobreakspace690--2};
\end{wideversion}
the techniques below do \emph{not} satisfy all the requirements of
\acro{ISO}~690--2, but they offer a solution that is at least
available to users of today's tools.

Until the new version of \BibTeX{} arrives, the simplest technique is
to use the \texttt{howpublished} field of the standard styles' \texttt{@misc}
function.  Of course, the strictures
about \Qref*{typesetting \acro{URL}s}{Q-setURL} still apply, so the
entry will look like:
\begin{quote}
\begin{verbatim}
@misc{...,
  ...,
  howpublished = "\url{http://...}"
}
\end{verbatim}
\end{quote}
A possible alternative approach is to use \BibTeX{} styles other than
the standard ones, that already have \acro{URL} entry types.
Candidates are:
\begin{itemize}
\item The \Package{natbib} styles (\Package{plainnat},
  \Package{unsrtnat} and \Package{abbrevnat}), which are extensions of
  the standard styles, principally for use with \Package{natbib}
  itself.  However, they've acquired \acro{URL}s and other ``modern''
  entries along the way.  The same author's \Package{custom-bib} is
  also capable of generating styles that honour \acro{URL} entries.
\item The \Package{babelbib} bundle, which offers % ! line break
  \Qref*{multilingual bibliographies}{Q-i18nbib}, similarly provides a
  set of standard-style equivalents that have \acro{URL} entries.
\item More modern styles such as the \Package{harvard} package (if the
  citation styles are otherwise satisfactory for you).
  \Package{Harvard} bibliography styles all include a ``\texttt{url}''
  field in their specification; however, the typesetting offered is
  somewhat feeble (though it does recognise and use
  \ProgName{LaTeX2HTML} macros if they are available, to create
  hyperlinks).
\end{itemize}
You can also acquire new \BibTeX{} styles by use of Norman Gray's
\Package{urlbst} system, which is based on a \ProgName{Perl} script
that edits an existing \BibTeX{} style file to produce a new
style. The new style thus generated has a \texttt{webpage} entry type, and
also offers support for \texttt{url} and \texttt{lastchecked} fields
in the other entry types.  The \ProgName{Perl} script comes with a set
of converted versions of the standard bibliography styles.

Another possibility is that some conventionally-published paper,
technical report (or even book) is also available on the Web.  In such
cases, a useful technique is something like:
\begin{quote}
\begin{wideversion}
\begin{verbatim}
@techreport{...,
  ...,
  note = "Also available as \url{http://...}"
}
\end{verbatim}
\end{wideversion}
\begin{narrowversion}
\begin{verbatim}
@techreport{...,
  ...,
  note = "Also available as 
                       \url{http://...}"
}
\end{verbatim}
\end{narrowversion}
\end{quote}
There is good reason to use the \Package{url} or \Package{hyperref}
packages in this context: \BibTeX{} has a habit of splitting
lines it considers excessively long, and if there are no space
characters for it to use as `natural' breakpoints, \BibTeX{} will
insert a comment (`\texttt{\textpercent{}}') character~\dots{}\@ which
is an acceptable character in an \acro{URL}.  Any current version of
either of the \Package{url} or \Package{hyperref} packages detects this
``\texttt{\textpercent{}}--end-of-line'' structure in its argument, and
removes it.
\begin{ctanrefs}
\item[babelbib \nothtml{\rmfamily}bundle]\CTANref{babelbib}
\item[custom-bib \nothtml{\rmfamily}bundle]\CTANref{custom-bib}
\item[harvard.sty]\CTANref{harvard}
\item[hyperref.sty]\CTANref{hyperref}
\item[natbib \nothtml{\rmfamily}styles]\CTANref{natbib}
\item[url.sty]\CTANref{url}
\item[urlbst]\CTANref{urlbst}
\end{ctanrefs}

\Question[Q-bibplain]{Using \BibTeX{} with \plaintex{}}

The file \File{btxmac.tex} (which is part of the \Eplain{} system)
contains macros and documentation for using \BibTeX{} with
\plaintex{}, either directly or with \Qref*{\Eplain{}}{Q-eplain}.  See
\Qref[question]{the use of \BibTeX{}}{Q-BibTeXing} for more
information about \BibTeX{} itself.
\begin{ctanrefs}
\item[btxmac.tex]\CTANref{btxmactex}
\item[eplain \nothtml{\rmfamily}system]\CTANref{eplain}
\end{ctanrefs}

\Question[Q-makebib]{Reconstructing \extension{bib} files}

Perhaps you've lost the \extension{bib} file you generated your document from,
or have been sent a document without one.  Or even, you've realised
the error of building a substantial document without the benefit of
\BibTeX{}\dots{}

The \ProgName{Perl} script, \Package{tex2bib} makes a reasonable job
of regenerating \extension{bib} files from \environment{thebibliography}
environments, provided that the original (whether automatically or
manually generated) doesn't deviate too far from the ``standard''
styles.

You are well-advised to check the output of the script.  While it will
not usually destroy information, it can quite reasonably mislabel it.

Documentation of the script is to be found in the file \File{tex2bib.readme}
\begin{ctanrefs}
\item[tex2bib]\CTANref{tex2bib}
\item[tex2bib.readme]\CTANref{tex2bib-doc}
\end{ctanrefs}

\Question[Q-bibprefixsort]{\BibTeX{} sorting and name prefixes}

\BibTeX{} recognises a bewildering array of name prefixes (mostly
those deriving from European language names); it ignores the prefixes
when sorting the bibliography~--- you want ``Ludwig van Beethoven''
sorted under ``Beethoven'', not under ``van''.  (Lamport made a witty
deliberate mistake with Beethoven's name, in the first edition of his
\LaTeX{} manual.)

However, a recurring issue is the desire to quote Lord Rayleigh's
publications (``Lord'' isn't an acceptable prefix), or names from
languages that weren't considered when \BibTeX{} was designed such as
``al-Wakil'' (transcribed from the Arabic).  What's needed is a
separate ``sort key'', but \BibTeX{} only allows such a thing in
citations of items that have no author or editor.

The solution is to embed the sort key in the author's name, but to
prevent it from being typeset.  Patashnik recommends a command
\csx{noopsort} (no-output-sortkey), which is defined and used as
follows:
\begin{quote}
\begin{wideversion}
\begin{verbatim}
@PREAMBLE{ {\providecommand{\noopsort}[1]{}} }
...
@ARTICLE{Rayleigh1,
AUTHOR = "{\noopsort{Rayleigh}}{Lord Rayleigh}",
...
}
\end{verbatim}
\end{wideversion}
\begin{narrowversion}
\begin{verbatim}
@PREAMBLE{
    {\providecommand{\noopsort}[1]{}}
}
...
@ARTICLE{Rayleigh1,
AUTHOR = 
   "\noopsort{Rayleigh}{Lord Rayleigh}",
...
}
\end{verbatim}
\end{narrowversion}
\end{quote}
Note that this \csx{noopsort} applies to the last name in this kind of
construct, so an author with an Arabic name might be rendered:
\begin{wideversion}
\begin{quote}
\begin{verbatim}
...
AUTHOR = "Ali {\noopsort{Hadiidii}}{al-Hadiidii}",
...
\end{verbatim}
\end{quote}
\end{wideversion}
\begin{narrowversion}
\begin{quote}
\begin{verbatim}
...
AUTHOR = 
   "Ali {\noopsort{Hadiidii}}{al-Hadiidii}",
...
\end{verbatim}
\end{quote}
\end{narrowversion}
A further use might deal with word order games, as in the famous
Vietnamese name:
\begin{wideversion}
\begin{quote}
\begin{verbatim}
...
AUTHOR = "\noopsort{Thanh Han The}{Han The Thanh}",
...
\end{verbatim}
\end{quote}
\end{wideversion}
\begin{narrowversion}
\begin{quote}
\begin{verbatim}
...
AUTHOR =
   "\noopsort{Thanh Han The}{Han The Thanh}",
...
\end{verbatim}
\end{quote}
\end{narrowversion}
though that author seems well-acquainted with Western confusion about
the significance of the parts of his name (even to the extent of
missing out the accentuation, as above\dots{}).

\Question[Q-bibtranscinit]{`Multi-letter' initials in \BibTeX{}}
\keywords{compound initials, transcribed initials}

If your bibliographic style uses initials~+ surname, you may encounter
a problem with some transcribed names (for example, Russian ones).
Consider the following example from the real world:
\begin{quote}
\begin{narrowversion}
\begin{verbatim}
@article{epifanov1997,
   author = {Epifanov, S. Yu. and
             Vigasin, A. A.},
   title  = ...
}
\end{verbatim}
\end{narrowversion}
\begin{wideversion}
\begin{verbatim}
@article{epifanov1997,
   author = {Epifanov, S. Yu. and Vigasin, A. A.},
   title  = ...
}
\end{verbatim}
\end{wideversion}
\end{quote}
Note that the ``Yu'' is the initial, not a complete name. However,
\BibTeX{}'s algorithms will leave you with a citation~--- 
slightly depending on the bibliographic style~--- that reads:
``S. Y. Epifanov and A. A. Vigasin, \dots{}''. instead of the intended
``S. Yu. Epifanov and A. A. Vigasin, \dots{}''.

One solution is to replace each affected initial by a command that 
prints the correct combination.  To keep your bibliography portable,
you need to add that command to your bibliography with the
\texttt{@preamble} directive:
\begin{quote}
\begin{narrowversion}
\begin{verbatim}
@preamble{ {\providecommand{\BIBYu}{Yu} } }

@article{epifanov1997,
   author   = {Epifanov, S. {\BIBYu}. and
               Vigasin, A. A.},
   title    = ...
}
\end{verbatim}
\end{narrowversion}
\begin{wideversion}
\begin{verbatim}
@preamble{ {\providecommand{\BIBYu}{Yu} } }

@article{epifanov1997,
   author   = {Epifanov, S. {\BIBYu}. and Vigasin, A. A.},
   title    = ...
}
\end{verbatim}
\end{wideversion}
\end{quote}
If you have many such commands, you may want to put them in a separate
file and \csx{input} that \LaTeX{} file in a \texttt{@preamble}
directive.

An alternative is to make the transcription look like an accent, from
\BibTeX{}'s point of view.  For this we need a control sequence that
does nothing:
\begin{wideversion}
\begin{quote}
\begin{verbatim}
@article{epifanov1997,
   author   = {Epifanov, S. {\relax Yu}. and Vigasin, A. A.},
   title    = ...
}
\end{verbatim}
\end{quote}
\end{wideversion}
\begin{narrowversion}
\begin{quote}
\begin{verbatim}
@article{epifanov1997,
   author   = {Epifanov, S. {\relax Yu}. and
               Vigasin, A. A.},
   title    = ...
}
\end{verbatim}
\end{quote}
\end{narrowversion}
Like the solution by generating extra commands, this involves tedious
extra typing; which of the two techniques is preferable for a given
bibliography will be determined by the names in it.  It should be
noted that a preamble that introduces lots of odd commands is usually
undesirable if the bibliography is a shared one.

``Compound'' initials (for single names made up of two or more words)
may be treated in the same way, so one can enter Forster's rather
complicated name as:
\begin{wideversion}
\begin{quote}
\begin{verbatim}
@article{forster2006,
  author   = {Forster, P.M. {\relax de F.} and Collins, M.},
  title    = ...
\end{verbatim}
\end{quote}
\end{wideversion}
\begin{narrowversion}
\begin{quote}
\begin{verbatim}
@article{forster2006,
  author   = {Forster, P.M. {\relax de F.} and
              Collins, M.},
  title    = ...
\end{verbatim}
\end{quote}
\end{narrowversion}
The same trick can be played if you're entering whole names:
\begin{quote}
\begin{verbatim}
...
  author   = {Epifanov, Sasha {\relax Yu}ri and
...
\end{verbatim}
\end{quote}
(though no guarantee, that either of those names is right, is
offered!)
However, if you're typing the names in the ``natural'' (Western) way,
with given names first, the trick:
\begin{quote}
\begin{verbatim}
...
  author   = {P.M. {\relax de F.} Forster and
...
\end{verbatim}
\end{quote}
doesn't work~--- ``de F. Forster'' is treated as a compound family
names.


\subsection{Creating citations}

\Question[Q-usebibtex]{``Normal'' use of \BibTeX{} from \LaTeX{}}

To create a bibliography for your document, you need to perform a
sequence of steps, some of which seem a bit odd.  If you choose to use
\BibTeX{}, the sequence is:

First: you need a \BibTeX{} bibliography file (a \extension{bib}
file)~--- see \Qref*[question]{``creating a \BibTeX{} file''}{Q-buildbib}.

Second: you must write your \LaTeX{} document to include a declaration
of the `style' of bibliography, citations, and a reference to the
bibliography file mentioned above.  So we may have a \LaTeX{}
file containing:
\begin{quote}
\begin{verbatim}
\bibliographystyle{plain}
...
Pooh is heroic~\cite{Milne:1926}.
...
Alice struggles~\cite{Carroll:1865}.
...
\bibliography{mybooks}
\end{verbatim}
\end{quote}
Note: we have bibliography style \Package{plain}, above, which is
nearly the simplest of the lot: a sample text, showing the sorts of
style choices available, can be found on Ken Turner's web site:
\URL{http://www.cs.stir.ac.uk/~kjt/software/latex/showbst.html}

Third: you must process the file.
\begin{quote}
\begin{verbatim}
latex myfile
\end{verbatim}
\end{quote}
As \LaTeX{} processes the file, the \csx{bibliographystyle} command
writes a note of the style to the \extension{aux} file; each
\csx{cite} command writes a note of the citation to the
\extension{aux} file, and the \csx{bibliography} command writes a note
of which \extension{bib} file is to be used, to the \extension{aux} file.

Note that, at this stage, \LaTeX{} isn't ``resolving'' any of the
citations: at every \csx{cite} command, \LaTeX{} will warn you of the
undefined citation, and when the document finishes, there will be a
further warning of undefined references.

%Some bibliography styles are designed to work with particular
%packages...

Fourth: you must run \BibTeX{}:
\begin{quote}
\begin{verbatim}
bibtex myfile
\end{verbatim}
\end{quote}
Don't try to tell \BibTeX{} anything but the file name: say
\texttt{bibtex myfile.aux} (because you know it's going to read the
\extension{aux} file) and \BibTeX{} will blindly attempt to process
\texttt{myfile.aux.aux}.

\BibTeX{} will scan the \extension{aux} file; it will find which bibliography
style it needs to use, and will ``compile'' that style; it will note
the citations; it will find which bibliography files it needs, and
will run through them matching citations to entries in the
bibliography; and finally it will sort the entries that have been
cited (if the bibliography style specifies that they should be
sorted), and outputs the resulting details to a \extension{bbl} file.

Fifth: you run \LaTeX{} again.  It warns, again, that each citation is
(still) undefined, but when it gets to the \csx{bibliography} command,
it finds a \extension{bbl} file, and reads it.  As it encounters each
\csx{bibitem} command in the file, it notes a definition of the
citation.

Sixth: you run \LaTeX{} yet again.  This time, it finds values for all
the citations, in its \extension{aux} file.  Other things being equal, you're
done\dots{} until you change the file.

If, while editing, you change any of the citations, or add new ones,
you need to go through the process above from steps 3 (first run of
\LaTeX{}) to 6, again, before the document is once again stable.
These four mandatory runs of \LaTeX{} make processing a document with
a bibliography even more tiresome than the normal two runs required to
resolve labels.

To summarise: processing to resolve citations requires: \LaTeX{};
\BibTeX{}; \LaTeX{}; \LaTeX{}.

\Question[Q-whatbst]{Choosing a bibliography style}

A large proportion of people are satisfied with one of Patashnik's
original ``standard'' styles, \Package{plain}, \Package{unsrt},
\Package{abbrv} and \Package{alpha}.  However, no style in that set
supports the ``author-date'' citation style that is popular in many
fields; but there are a very large number of contributed styles
available, that \emph{do} support the format.

(Note that author-date styles arose because the simple and clear
citation style that \Package{plain} produces is so awkward in a
traditional manuscript preparation scenario.  However, \TeX{}-based
document production does away with all those difficulties, leaving us
free once again to use the simple option.)

Fortunately, help is at hand, on the Web, with this problem:
\begin{itemize}
\item a sample text, showing the sorts of style choices available, can
  be found on 
\begin{narrowversion}
  Ken Turner's web site:
  \URL{http://www.cs.stir.ac.uk/~kjt/software/latex/showbst.html};
\end{narrowversion}
\begin{wideversion}
  % ! line break
  \href{http://www.cs.stir.ac.uk/~kjt/software/latex/showbst.html}{Ken Turner's web site};
\end{wideversion}
\item an excellent survey, that lists a huge variety of styles,
  sorted into their nominal topics as well as providing a good range
  of examples, is the Reed College % ! line break
\begin{narrowversion}
  ``Choosing a \BibTeX{} style''
  (\URL{http://web.reed.edu/cis/help/LaTeX/bibtexstyles.html}).
\end{narrowversion}
\begin{wideversion}
  % ! line break
  ``\href{http://web.reed.edu/cis/help/LaTeX/bibtexstyles.html}{Choosing a \BibTeX{} style}''.
\end{wideversion}
\end{itemize}

Of course, these pages don't cover everything; the problem the
inquisitive user faces, in fact, is to find what the various available
styles actually do.  This is best achieved (if the links above don't
help) by using \Package{xampl.bib} from the \BibTeX{} documentation
distribution: one can get a pretty good feel for any style one has to
hand using this ``standard'' bibliography.  For style
\Package{my-style.bst}, the simple \LaTeX{} document:
\begin{quote}
\begin{verbatim}
\documentclass{article}
\begin{document}
\bibliographystyle{my-style}
\nocite{*}
\bibliography{xampl}
\end{document}
\end{verbatim}
\end{quote}
will produce a representative sample of the citations the style will
produce.  (Because \Package{xampl.bib} is so extreme in some of its
``examples'', the \BibTeX{} run will also give you an interesting
selection of \BibTeX{}'s error messages\dots{})
\begin{ctanrefs}
\item[xampl.bib]\CTANref{xampl-bib}
\end{ctanrefs}

\Question[Q-chapbib]{Separate bibliographies per chapter?}

A separate bibliography for each `chapter' of a document can be provided
with the package \Package{chapterbib} (which comes with a bunch of
other good bibliographic things).  The package allows you a
different bibliography for each \csx{include}d file (i.e., despite the
package's name, the availability of bibliographies is related to the
component source files of the document rather than to the chapters that
logically structure the document).

The package \Package{bibunits} ties bibliographies to logical units
within the document: the package will deal with chapters and sections
(as defined by \LaTeX{} itself) and also defines a \environment{bibunit}
environment so that users can select their own structuring.

\Qref*{The \Package{biblatex} package}{Q-biblatex}, with
\ProgName{biber}, provides a similar facility; enclose the text for
which you want a local bibliography in a \environment{refsection}
environment, and place a \csx{printbibliography} command as the last
thing in that environment:
\begin{quote}
\begin{verbatim}
\begin{refsection}
\chapter{First chapter}
\section{Foo}
Some text \cite{this}
with citations \cite{that}.
\printbibliography
\end{refsection}
\end{verbatim}
\end{quote}
Then process with \latex{} (of whatever flavour) and use
\ProgName{biber} to process the bibliography output.  Note that
\csx{printbibliography} can take an optional argument
\pkgoption{heading=bib title} to provide the bibliography with a
(sub)section title.
\begin{ctanrefs}
\item[biber]\CTANref{biber}
\item[biblatex]\CTANref{biblatex}
\item[bibunits.sty]\CTANref{bibunits}
\item[chapterbib.sty]distributed as part of \CTANref{cite}[chapterbib]
\end{ctanrefs}
\LastEdit{2013-01-04}

\Question[Q-multbib]{Multiple bibliographies?}

If you're thinking of multiple bibliographies tied to some part of
your document (such as the chapters within the document), please see
\Qref[question]{bibliographies per chapter}{Q-chapbib}.

For more than one bibliography, there are three options.

The \Package{multibbl} package offers a very simple interface: you use
a command \csx{newbibliography} to define a bibliography ``tag''.  The package
redefines the other bibliography commands so that each time you use any one
of them, you give it the tag for the bibliography where you want the
citations to appear.  The \csx{bibliography} command itself also takes
a further extra argument that says what title to use for the resulting
section or chapter (i.e., it patches
\nothtml{\csx{refname} and \csx{bibname}~---}
\Qref{\csx{refname} and \csx{bibname}}{Q-fixnam}\nothtml{~---} in a
\Package{babel}-safe way).  So one might write:
\begin{narrowversion}
\begin{quote}
\begin{verbatim}
\usepackage{multibbl}
\newbibliography{bk}
\bibliographystyle{bk}{alpha}
\newbibliography{art}
\bibliographystyle{art}{plain}
...
\cite[pp.~23--25]{bk}{milne:pooh-corner}
...
\cite{art}{einstein:1905}
...
\bibliography{bk}{book-bib}%
             {References to books}
\bibliography{art}{art-bib}%
             {References to articles}
\end{verbatim}
\end{quote}
\end{narrowversion}
\begin{wideversion}
\begin{quote}
\begin{verbatim}
\usepackage{multibbl}
\newbibliography{bk}
\bibliographystyle{bk}{alpha}
\newbibliography{art}
\bibliographystyle{art}{plain}
...
\cite[pp.~23--25]{bk}{milne:pooh-corner}
...
\cite{art}{einstein:1905}
...
\bibliography{bk}{book-bib}{References to books}
\bibliography{art}{art-bib}{References to articles}
\end{verbatim}
\end{quote}
\end{wideversion}
(Note that the optional argument of \csx{cite} appears \emph{before} the
new tag argument, and that the \csx{bibliography} commands may list
more than one \extension{bib} file~--- indeed all \csx{bibliography} commands
may list the same set of files.)

The \csx{bibliography} data goes into files whose names are
\meta{tag-name}\emph{.aux}, so you will need to run
\begin{quote}
\begin{verbatim}
bibtex bk
bibtex art
\end{verbatim}
\end{quote}
after the first run of \LaTeX{}, to get the citations in the correct
place.

The \Package{multibib} package allows you to define a series of
``additional topics'', each of which comes with its own series of
bibliography commands.  So one might write:
\begin{quote}
\begin{verbatim}
\usepackage{multibib}
\newcites{bk,art}%
         {References from books,%
          References from articles}
\bibliographystylebk{alpha}
\bibliographystyleart{plain}
...
\citebk[pp.~23--25]{milne:pooh-corner}
...
\citeart{einstein:1905}
...
\bibliographybk{book-bib}
\bibliographyart{art-bib}
\end{verbatim}
\end{quote}
Again, as for \Package{multibbl}, any \csx{bibliography...} command may
scan any list of \extension{bib} files.

\BibTeX{} processing with \Package{multibib} is much like that with
\Package{multibbl}; with the above example, one needs:
\begin{quote}
\begin{verbatim}
bibtex bk
bibtex art
\end{verbatim}
\end{quote}
Note that, unlike \Package{multibbl}, \Package{multibib} allows a
simple, unmodified bibliography (as well as the ``topic'' ones).  

The \Package{bibtopic} package allows you separately to cite several
different bibliographies.  At the appropriate place in your document,
you put a sequence of \environment{btSect} environments (each of which
specifies a bibliography database to scan) to typeset the separate
bibliographies.  Thus, one might have a file \File{diss.tex} containing:
\begin{quote}
\begin{verbatim}
\usepackage{bibtopic}
\bibliographystyle{alpha}
...
\cite[pp.~23--25]{milne:pooh-corner}
...
\cite{einstein:1905}
...
\begin{btSect}{book-bib}
\section{References from books}
\btPrintCited
\end{btSect}
\begin{btSect}[plain]{art-bib}
\section{References from articles}
\btPrintCited
\end{btSect}
\end{verbatim}
\end{quote}
Note the different way of specifying a bibliographystyle: if you want
a different style for a particular bibliography, you may give it as an
optional argument to the \environment{btSect} environment.

Processing with \BibTeX{}, in this case, uses \extension{aux} files whose names
are derived from the name of the base document.  So in this example
you need to say:
\begin{quote}
\begin{verbatim}
bibtex diss1
bibtex diss2
\end{verbatim}
\end{quote}

There is also a command \csx{btPrintNotCited}, which gives the rest of
the content of the database (if nothing has been cited from the
database, this is equivalent to \LaTeX{} standard \cmdinvoke{nocite}{*}).

However, the \emph{real} difference from \Package{multibbl} and
\Package{multibib} is that selection of what appears in each
bibliography section is determined in \Package{bibtopic} by what's in
the \extension{bib} files.

An entirely different approach is taken by the \Package{splitbib}
package.  You provide a \environment{category} environment, in the
preamble of your document, for each category you want a separate
citation list for.  In each environment, you list the \csx{cite} keys
that you want listed in each category.  The \csx{bibliography} command
(or, more precisely, the \environment{thebibliography} environment it
uses) will sort the keys as requested.  (Keys not mentioned in a
\environment{category} appear in a ``misc'' category created in the
sorting process.)  A code example appears in the package documentation
(a \acro{PDF} file in the \acro{CTAN} directory,
\begin{htmlversion}
  which you can browse to, from the link, below).
\end{htmlversion}
\htmlignore
  see the file list, below).
\endhtmlignore
\begin{ctanrefs}
\item[bibtopic.sty]\CTANref{bibtopic}
\item[multibbl.sty]\CTANref{multibbl}
\item[multibib.sty]\CTANref{multibib}
\item[splitbib.sty]\CTANref{splitbib}
\end{ctanrefs}


\Question[Q-bibinline]{Putting bibliography entries in text}

This is a common requirement for journals and other publications in
the humanities.  Sometimes the requirement is for the entry to appear
in the running text of the document, while other styles require that
the entry appear in a footnote.

Options for entries in running text are
\begin{itemize}
\item The package \Package{bibentry}, which puts slight restrictions
  on the format of entry that your \extension{bst} file generates, but is
  otherwise undemanding of the bibliography style.
\item The package \Package{inlinebib}, which requires that you use its
  \File{inlinebib.bst}.  \Package{Inlinebib} was actually designed for
  footnote citations: its \emph{expected} use is that you place a
  citation inline as the argument of a \csx{footnote} command.
\item The package \Package{jurabib}, which was originally designed for
  German law documents, and has comprehensive facilities for the
  manipulation of citations.  The package comes with four bibliography
  styles that you may use: \File{jurabib.bst}, \File{jhuman.bst} and
  two Chicago-like ones.
\end{itemize}

Options for entries in footnotes are
\begin{itemize}
\item The package \Package{footbib}, and
\item Packages \Package{jurabib} and \Package{inlinebib}, again.
\end{itemize}
Note that \Package{jurabib} does the job using \LaTeX{}'s standard
footnotes, whereas \Package{footbib} creates its own sequence of
footnotes.  Therefore, in a document which has other footnotes, it may
be advisable to use \Package{jurabib} (or of course
\Package{inlinebib}), to avoid confusion of footnotes and
foot-citations.

The \Package{usebib} package offers a `toolbox', which allows the user
to place exactly what is needed, in the text (that is, rather than a
full citation).  The package's command, that does the actual
typesetting, is \cmdinvoke{usebibdata}{\meta{key}}{\meta{field}}; it
typesets the \emph{field} item from the entry \emph{key} in the
bibliography; the user then formats the entry as desired~--- obviously
one could construct one's own bibliography, altogether, from this
command, but it would quickly become tedious.
\begin{ctanrefs}
\item[bibentry.sty]\emph{Distributed with} \CTANref{natbib}
\item[footbib.sty]\CTANref{footbib}
\item[inlinebib.sty]\CTANref{inlinebib}
\item[jurabib.sty]\CTANref{jurabib}
\item[usebib.sty]\CTANref{usebib}
\end{ctanrefs}
\LastEdit{2012-03-23}

\Question[Q-citesort]{Sorting and compressing citations}

If you give \LaTeX{}
\cmdinvoke{cite}{fred,joe,harry,min}, its default commands could give
something like ``[2,6,4,3]'';
this looks awful.  One can of course get the things in order by
rearranging the keys in the \csx{cite} command, but who wants to do
that sort of thing for no more improvement than ``[2,3,4,6]''?

The \Package{cite} package sorts the numbers and detects consecutive
sequences, so creating ``[2--4,6]''.  The \Package{natbib} package,
with the \pkgoption{numbers} and \pkgoption{sort\&compress} options, will
do the same when working with its own numeric bibliography styles
(\File{plainnat.bst} and \File{unsrtnat.bst}).

The package \Package{biblatex} has a built-in style
\Package{numeric-comp} for its bibliographies.
\begin{ctanrefs}
\item[biblatex.sty]\CTANref{biblatex}
\item[cite.sty]\CTANref{cite}
\item[hypernat.sty]\CTANref{hypernat}
\item[hyperref.sty]\CTANref{hyperref}
\item[plainnat.bst]Distributed with \CTANref{natbib}
\item[unsrtnat.bst]Distributed with \CTANref{natbib}
\end{ctanrefs}
\LastEdit{2012-02-10}

\Question[Q-mcite]{Multiple citations}

A convention sometimes used in physics journals is to ``collapse'' a group of
related citations into a single entry in the bibliography.  \BibTeX{},
by default, can't cope with this arrangement, but the \Package{mcite}
and \Package{mciteplus} packages deal with the problem.

\Package{mcite} overloads the \csx{cite} command to recognise a
``\texttt{*}'' at the start of a key, so that citations of the form
\begin{quote}
\begin{verbatim}
\cite{paper1,*paper2}
\end{verbatim}
\end{quote}
appear in the document as a single citation, and appear arranged
appropriately in the bibliography itself.  You're not limited to
collapsing just two references.  You can mix ``collapsed'' references
with ``ordinary'' ones, as in
\begin{quote}
\begin{verbatim}
\cite{paper0,paper1,*paper2,paper3}
\end{verbatim}
\end{quote}
Which will appear in the document as 3~citations ``[4,7,11]''
(say)~--- citation `4' will refer to paper~0, `7' will refer to a
combined entry for paper~1 and paper~2, and `11' will refer to
paper~3.

You need to make a small change to the bibliography style
(\extension{bst}) file you use; the \Package{mcite} package
documentation tells you how to do that.

Most recent versions of \Class{REVTeX} (version 4.1 and later), in
conjunction with recent versions of \Package{natbib}, already contain
support for combined citations and so no longer even need
\Package{mciteplus} (but \Package{mciteplus} is more general and will
work with many other class and package combinations).

The \Package{mciteplus} package adresses many of the infelicites of
\Package{mcite}.  Again, `ordinary' \extension{bst} files will not
work with \Package{mciteplus}, but the package documentation explains
how to patch an existing \BibTeX{} style.

The \Package{collref} package takes a rather different approach to the
problem, and will work with most (if not all) \BibTeX{} packages.
\Package{Collref} spots common subsets of the references, so if it
sees a sequence
\begin{quote}
\begin{verbatim}
\cite{paper0,paper1,paper2,paper3}
...
\cite{some_other_paper,paper1,paper2,and_another}
\end{verbatim}
\end{quote}
it will collect \texttt{paper1} and \texttt{paper2} as a multiple reference.
\begin{ctanrefs}
\item[collref.sty]\CTANref{collref}
\item[mcite.sty]\CTANref{mcite}
\item[mciteplus.sty]\CTANref{mciteplus}
\item[natbib.sty]\CTANref{natbib}
\item[revtex 4.1]\CTANref{revtex4-1}
\end{ctanrefs}
\LastEdit{2010-07-07}

\Question[Q-backref]{References from the bibliography to the citation}

A link (or at least a page reference), from the bibliography to the
citing command, is often useful in large documents.

Two packages support this requirement, \Package{backref} and
\Package{citeref}.  \Package{Backref} is part of the
\Package{hyperref} bundle, and supports hyperlinks back to the citing
command.

\Package{Citeref} is the older, and seems to rely on rather simpler
(and therefore possibly more stable) code; it produces a list of page
references, only.  It doesn't interact well with other citation
packages (for example, \Package{cite}), which probably reflects its
antiquity (it's derived from a \LaTeXo{} package).

Neither collapses
lists of pages (``\texttt{5, 6, 7}'' comes out as such, rather than as
``\texttt{5--7}''), but neither package repeats the reference to a page that
holds multiple citations.  (The failure to collapse lists is of course
forgiveable in the case of the \Package{hyperref}-related
\Package{backref}, since the concept of multiple hyperlinks from the
same anchor is less than appealing.)
\begin{ctanrefs}
\item[backref.sty]Distributed with \CTANref{hyperref}
\item[citeref.sty]\CTANref{citeref}
\end{ctanrefs}

\Question[Q-sortbib]{Sorting lists of citations}

\BibTeX{} has a sorting function, and most \BibTeX{} styles sort the
citation list they produce; most people find this desirable.

However, it is perfectly possible to write a
\environment{thebibliography} environment that \emph{looks} as if it
came from \BibTeX{}, and many people do so (in order to save time in
the short term).

The problem arises when \environment{thebibliography}-writers decide
their citations need to be sorted.  A common misapprehension is to
insert \cmdinvoke{bibliographystyle}{alpha} (or similar) and expect
the typeset output to be sorted in some magical way.  \BibTeX{}
doesn't work that way!~--- if you write \environment{thebibliography},
you get to sort its contents.  \BibTeX{} will only sort the contents
of a \environment{thebibliography} environment when it creates it, to
be inserted from a \extension{bbl} file by a \csx{bibliography} command.

\Question[Q-compactbib]{Reducing spacing in the bibliography}

Bibliographies are, in fact, implemented as lists, so all the
confusion about \Qref*{reducing list item spacing}{Q-complist} also
applies to bibliographies.

If the \Package{natbib} package `works' for you (it may not if you are using
some special-purpose bibliography style), the solution is relatively
simple~--- add
\begin{quote}
\begin{verbatim}
\usepackage{natbib}
\setlength{\bibsep}{0.0pt}
\end{verbatim}
\end{quote}
to the preamble of your document.

The \Package{compactbib} package has a similar effect.  Its primary
purpose is to produce two bibliographies, and it seems to preclude use
of \BibTeX{} (though the package documentation, in the package file
itself, isn't particularly clear).

Otherwise, one is into unseemly hacking of something or other.  The
\Package{mdwlist} package actually does the job, but it doesn't work
here, because it makes a different-named list, while the name
``\environment{thebibliography}'' is built into \LaTeX{} and
\BibTeX{}.  Therefore, we need to % ! line break
\Qref*{patch the underlying macro}{Q-patch}:
\begin{quote}
\begin{verbatim}
\let\oldbibliography\thebibliography
\renewcommand{\thebibliography}[1]{%
  \oldbibliography{#1}%
  \setlength{\itemsep}{0pt}%
}
\end{verbatim}
\end{quote}
The \Package{savetrees} package performs such a patch, among a
plethora of space-saving measures: you can, in principle, suppress all
its other actions, and have it provide you a compressed bibliography
\emph{only}.
\begin{ctanrefs}
\item[compactbib.sty]\CTANref{compactbib}
\item[mdwlist.sty]Distributed as part of \CTANref{mdwtools}[mdwlist]
\item[natbib.sty]\CTANref{natbib}
\item[savetrees.sty]\CTANref{savetrees}
\end{ctanrefs}

\Question[Q-bibtocorder]{Table of contents rearranges ``\Package{unsrt}'' ordering}

If you're using the \Package{unsrt} bibliography style, you're
expecting that your bibliography will \emph{not} be sorted, but that
the entries will appear in the order that they first appeared in your
document.

However, if you're unfortunate enough to need a citation in a section
title, and you also have a table of contents, the citations that now
appear in the table of contents will upset the ``natural'' ordering
produced by the \Package{unsrt} style.  Similarly, if you have
citations in captions, and have a list of figures (or tables).

There's a pretty simple ``manual'' method for dealing with the
problem~--- when you have the document stable:
\begin{enumerate}
\item Delete the \extension{aux} file, and any of \extension{toc},
  \extension{lof}, \extension{lot} files.
\item Run \LaTeX{}.
\item Run \BibTeX{} for the last time.
\item Run \LaTeX{} often enough that things are stable again.
\end{enumerate}
Which is indeed simple, but it's going to get tedious when you've
found errors in your ``stable'' version, often enough.

The package \Package{notoccite} avoids the kerfuffle, and suppresses
citations while in the table of contents, or lists of figures, tables
(or other floating things: the code is quite general).
\begin{ctanrefs}
\item[notoccite.sty]\CTANref{notoccite}
\end{ctanrefs}

\Question[Q-i18nbib]{Non-english bibliographies}

Like so much of early \AllTeX{} software, \BibTeX{}'s assumptions were
firmly rooted in what its author knew well, viz., academic papers in
English (particularly those with a mathematical bent).  \BibTeX{}'s
standard styles all address exactly that problem, leaving the user who
writes in another language (or who deal with citations in the style of
other disciplines than maths) to strike out into contributed software.

For the user whose language is not English, there are several
alternatives.  Possibly most straightforward is to switch to using
\Package{biblatex}, which can produce a bibliography appropriate to
several languages.  However, \Package{biblatex} is large and has
correspondingly large documentation (though it is well-written and
pleasingly typeset), so its adoption takes time.

Otherwise, the simplest procedure is to provide translations of
\BibTeX{} styles into the
required language: the solitary \Package{finplain.bst} does that for
Finnish; others one can find are for Danish (\Package{dk-bib}), French
(\Package{bib-fr}), German (\Package{bibgerm}), Norwegian
(\Package{norbib}) and Swedish (\Package{swebib}) bundles (of which
the \Package{bib-fr} set is the most extensive).  The \Package{spain}
style implements a traditional Spanish citation style.

These static approaches solve the problem, for the languages that have
been covered by them.  Unfortunately, with such an approach, a lot of
work is needed for every language involved.  Two routes to a solution
of the ``general'' problem are available~--- that offered by
\Package{babelbib}, and the % ! line break
\Qref*{\Package{custom-bib} mechanism for generating styles}{Q-custbib}.

\Package{Babelbib} (which is a development of the ideas of the
\Package{bibgerm} package) co-operates with \Package{babel} to control
the language of presentation of citations (potentially at the level of
individual items).  The package has a built-in set of languages it
`knows about', but the documentation includes instructions on defining
commands for other languages.  \Package{Babelbib} comes with its own
set of bibliography styles, which could be a restriction if there
wasn't also a link from \Package{custom-bib}.

The \Package{makebst} menu of \Package{custom-bib} allows you to
choose a language for the \BibTeX{} style you're generating (there are
14 languages to choose; it looks as if \Package{spain.bst}, mentioned
above, was generated this way).  If, however, you opt not to specify a
language, you are asked whether you want the style to interact with
\Package{babelbib}; if you do so, you're getting the best of both
worlds~--- formatting freedom from \Package{custom-bib} and linguistic
freedom via the extensibility of \Package{babelbib}
\begin{ctanrefs}
\item[babelbib.sty]\CTANref{babelbib}
\item[bib-fr \nothtml{\rmfamily\upshape}bundle]\CTANref{bib-fr}
\item[bibgerm \nothtml{\rmfamily\upshape}bundle]\CTANref{bibgerm}
\item[biblatex.sty]\CTANref{biblatex}
\item[custom-bib \nothtml{\rmfamily\upshape}bundle]\CTANref{custom-bib}
\item[finplain.bst]\CTANref{finplain}
\item[norbib \nothtml{\rmfamily\upshape}bundle]\CTANref{norbib}
\item[spain]\CTANref{spain}
\item[swebib \nothtml{\rmfamily\upshape}bundle]\CTANref{swebib}
\end{ctanrefs}
\LastEdit{2012-03-23}

\Question[Q-formbiblabel]{Format of numbers in the bibliography}

By default, \LaTeX{} makes entries in the bibliography look like:
\begin{quote}
  [1] Doe, Joe et al.  Some journal.  2004.\\ {}
  [2] Doe, Jane et al. Some journal. 2003.
\end{quote}
while many documents need something like:
\begin{quote}
  1. Doe, Joe et al.  Some journal.  2004.\\
  2. Doe, Jane et al. Some journal. 2003.
\end{quote}

This sort of change may be achieved by many of the ``general''
citation packages; for example, in \Package{natbib}, it's as simple as:
\begin{quote}
\begin{verbatim}
\renewcommand{\bibnumfmt}[1]{#1.}
\end{verbatim}
\end{quote}
but if you're not using such a package, the following internal
\LaTeX{} commands, in the preamble of your document, will do the job:
\begin{quote}
\begin{verbatim}
\makeatletter
\renewcommand*{\@biblabel}[1]{\hfill#1.}
\makeatother
\end{verbatim}
\end{quote}
\begin{ctanrefs}
\item[natbib.sty]\CTANref{natbib}
\end{ctanrefs}

\subsection{Manipulating whole bibliographies}

\Question[Q-nocite*]{Listing all your \BibTeX{} entries}

\LaTeX{} and \BibTeX{} co-operate to offer special treatment of this
requirement.  The command \cmdinvoke{nocite}{*} is specially treated,
and causes \BibTeX{} to generate bibliography entries for every entry
in each \extension{bib} file listed in your \csx{bibliography} statement, so
that after a \LaTeX{}--\BibTeX{}--\LaTeX{} sequence, you have a
document with the whole thing listed.

Note that \LaTeX{} \emph{doesn't} produce
% beware line breaks
``\texttt{Citation ... undefined}'' or
``\texttt{There were undefined references}'' warnings in respect of
\cmdinvoke{nocite}{*}.  This isn't a problem if you're running
\LaTeX{} ``by hand'' (you \emph{know} exactly how many times you have
to run things), but the lack might confuse automatic processors that
scan the log file to determine whether another run is necessary.

A couple of packages are available, that aim to reduce the impact of
\cmdinvoke{nocite}{*} of a large citation database.  \Package{Biblist}
was written for use under \LaTeXo{}, but seems to work well enough;
\Package{listbib} is more modern.  Both provide their own
\extension{bst} files.  (The impact of large databases was significant
in the old days of \LaTeX{} systems with very little free memory; this
problem is less significant now than it once was.)
\begin{ctanrefs}
\item[biblist.sty]\CTANref{biblist}
\item[listbib.sty]\CTANref{listbib}
\end{ctanrefs}

\Question[Q-htmlbib]{Making \acro{HTML} of your Bibliography}

A neat solution is offered by the \Package{noTeX} bibliography style.
This style produces a \extension{bbl} file which is in fact a series of
\acro{HTML} `\texttt{P}' elements of class \texttt{noTeX}, and which
may therefore be included in an \acro{HTML} file.  Provision is made
for customising your bibliography so that its content when processed by
\Package{noTeX} is different from that presented when it is processed
in the ordinary way by \AllTeX{}.

A thorough solution is offered by \Package{bib2xhtml}; using it, you
make use of one of its modified versions of many common \BibTeX{}
styles, and post-process the output so produced using a
\ProgName{perl} script.

A more conventional translator is the \ProgName{awk} script
\ProgName{bbl2html}, which translates the \extension{bbl} file you've generated:
a sample of the script's output may be viewed on the web, at
\URL{http://rikblok.cjb.net/lib/refs.html}
\begin{ctanrefs}
\item[bbl2html.awk]\CTANref{bbl2html}
\item[bib2xhtml]\CTANref{bib2xhtml}
\item[noTeX.bst]\CTANref{noTeX}
\end{ctanrefs}

