% $Id: faq-getit.tex,v 1.11 2014/01/22 17:29:03 rf10 Exp $

\section{Acquiring the Software}

\Question[Q-archives]{Repositories of \TeX{} material}

To aid the archiving and retrieval of of \TeX{}-related files, a
\acro{TUG} working group developed the Comprehensive \TeX{} Archive
Network (\acro{CTAN}).  Each \acro{CTAN} site has identical material,
and maintains authoritative versions of its material.  These
collections are extensive; in particular, almost everything mentioned
in this \acro{FAQ}
is archived at the \acro{CTAN} sites (see the lists of software at the
end of each answer).

There are two main \acro{CTAN} sites (in Germany and the \acro{UK}),
but users should ordinarily collect material via the % ! line break
\href{http://mirror.ctan.org/}{\acro{CTAN} mirror redirector}
(which connects you to a mirror that is in some sense ``near'' to
you).

To access a particular thing through the \texttt{mirror.ctan.org}
mechanism, simply place the \acro{CTAN} path after the base
\acro{URL}; so
\URL{http://mirror.ctan.org/macros/latex/contrib/footmisc/} will
connect you to the \Package{footmisc} directory at some \acro{CTAN}
mirror.

For details of how to find files at \acro{CTAN} sites, see
``\Qref*{finding \AllTeX{} files}{Q-findfiles}''.

The \TeX{} user who has no access to any sort of
network may buy a copy of the archive as part of the
\Qref*{\texlive{} distribution}{Q-CD}; the disc is, necessarily, out
of date, but it is likely to be better than what (if anything) came
with your operating system.
\LastEdit{2012-12-14}

\Question[Q-tds-zip]{Ready-built installation files on the archive}

The \Qref*{\acro{TDS}}{Q-tds} is a simple structure, and almost all
files can be installed simply by putting them in the ``right'' place,
and updating a single index.  (Note, this simple idea typically
doesn't work for fonts, unless they're distributed as \MF{} source.)

The CTAN network is therefore acquiring ``\acro{TDS}-\acro{ZIP}'' files,
which have a built-in directory structure that matches the \acro{TDS}.
These files have to be built, and the \acro{CTAN} team has asked that
package authors supply them (the team will advise, of course, if the
author has trouble).  The \acro{CTAN} team hopes that the extra work
involved will contribute to happier lives for package users, which in
turn must surely help keep the TeX community lively and active.

At the time of writing, there are rather few \extension{tds.zip}
files (by comparison with the huge number of packages that are
available).  As packages are updated, the number of files is
steadily increasing, but it will be a long time before the whole set
is covered.

Use of the files is discussed in % ! line break
``\Qref*{installing using ready-built \acro{ZIP} files}{Q-inst-tds-zip}''.

\Question[Q-nonfree]{What was the \acro{CTAN} \texttt{nonfree} tree?}

When \acro{CTAN} was founded, in the 1990s, it was unusual to publish
the terms under which a \TeX{}-related package was distributed (or, at
any rate, to publish those terms formally).

With the advent of the \TeX{} \emph{distributions}, however, people
started to realise the need for such information, to protect those who
create, distribute or sell the discs that hold the packages, etc.
With the licence information available, the distributors can decide
which packages may be distributed.

The \acro{CTAN} team decided that it would be useful for users (and
distributors, not to say package authors) to separate packages that
were candidates for distribution, and those that were in some sense
``not free''.  Thus was the \texttt{nonfree} tree born.

From the start, the \texttt{nonfree} tree was controversial: the terms
under which a package would be placed on the tree were hotly
contested, and the \acro{CTAN} team were only able slowly to populate
the tree.  It became obvious to the team that the project would never
have been completed.

The \acro{CTAN} catalogue now records the nature of the licences of a
good proportion of the packages it describes (though there remain
several for which the licence is unknown, which is as good, for the
distributors, as a licence forbidding distribution).  Since the
catalogue's coverage of \acro{CTAN} is good (and slowly improving),
the general rule for distributors has become
\begin{quote}
``if the package is listed in the catalogue, check there to see
whether you should distribute; if the package is not listed in the
catalogue, don't think of distributing it''.
\end{quote}
(The catalogue only has a modest % ! line break
\CTANhref{cat-licences}{list of licences}, but it covers the set used
by packages on \acro{CTAN}, with a wild-card ``\texttt{other-free}''
which covers packages that the \acro{CTAN} administrators believe to
be free even though the authors haven't used a standard licence.)

There is a corollary to the `general rule': if you notice something
that ought to be in the distributions, for which there is no catalogue
entry, please let the \acro{CTAN} team (\mailto{ctan@dante.de}) know.
It may well be that the package has simply been missed, but some aren't
catalogued because there's no documentation and the team just doesn't
understand the package.

In the light of the above, the \texttt{nonfree} tree is being
dismantled, and its contents moved (or moved \emph{back}) to the main
\acro{CTAN} tree.  So the answer to the question is, now, ``the
nonfree tree was a part of \acro{CTAN}, whose contents are now in
the main tree''.

\Question[Q-uploads]{Contributing a file to the archives}

You have something to submit to the archive~--- good news!

Before we even start, here's a check-list of things to sort out:
\begin{enumerate}
\item Licence: in the spirit of \TeX{}, we hope for free software; in
  the spirit of today's lawyer-enthralled society, \acro{CTAN}
  provides a % ! line break
  \href{http://mirror.ctan.org/help/Catalogue/licenses.html}{list of ``standard'' licence statements}.
  Make sure that there's a formal statement of the licence of your
  package, somewhere in the files you upload; beyond the \acro{CTAN}
  installation, your package is a candidate for inclusion in \AllTeX{}
  distributions~\dots{} and thereafter, also in operating system
  distributions~\dots{} and the people who bundle all these things up
  need a clear statement of your intent. 
\item Documentation: it's good for users to be able to browse
  documentation before downloading a package.  You need at least a
  plain text \File{README} file (exactly that name, upper case and no
  \extension{txt} extension); in addition a
  \acro{PDF} file of the package documentation, prepared for screen
  reading, is highly desirable.
\item Name: endless confusion is caused by name clashes.  If your
  package has the same name as one already on \acro{CTAN}, or if your
  package installation generates files of the same name as something
  in a ``normal'' distribution, the \acro{CTAN} team will delay
  installation while they check that you're doing the right thing:
  they may nag you to change the name, or to negotiate a take-over
  with the author of the original package. % beware line break
  \Qref*{Browse the archive}{Q-findfiles} to ensure uniqueness.
  
  The name you choose should also (as far as possible) be somewhat
  descriptive of what your submission actually \emph{does}; while
  ``descriptiveness'' is to some extent in the eye of the beholder,
  it's clear that names such as \File{mypackage} or \File{jiffy}
  aren't suitable.
\end{enumerate}
You upload via the
\begin{flatversion}
  CTAN upload redirector: \URL{http://www.ctan.org/upload}
\end{flatversion}
\begin{hyperversion}
  \href{http://www.ctan.org/upload}{CTAN upload redirector}
\end{hyperversion}
(the archive's main page has a link).  The upload page shows what it
needs to know, and allows you to enter the information.  The mechanism
can only accept one file per upload: if you had intended to upload
lots of files, you need to bundle them into an `archive' file of some
sort; acceptable formats are \extension{zip} and \extension{tar.gz}
(most uploads are packed in \extension{zip} format).  Once you have
completed your upload, the redirector assigns it to a member of the
team for processing.

If you can't use this method, or if you find something confusing, ask
advice of the
\begin{flatversion}
  \acro{CTAN} management team (\mailto{ctan@dante.de}).
\end{flatversion}
\begin{hyperversion}
  \href{mailto:ctan@dante.de}{\acro{CTAN} management team}
\end{hyperversion}

If your package is large, or regularly updated, it may be appropriate
to `mirror' your contribution direct into \acro{CTAN}.
Mirroring is only practical using \texttt{ftp} or \texttt{rsync}, so
this facility is limited to packages offered by a server that uses one
of those protocols.
\begin{ctanrefs}
\item[README.uploads]\CTANref{CTAN-uploads}
\end{ctanrefs}
\LastEdit{2013-01-31}

\Question[Q-findfiles]{Finding \AllTeX{} files}

Modern \TeX{} distributions contain a huge array of various sorts of
files, but sooner or later most people need to find something that's
not in their present system (if nothing else, because they've heard
that something has been updated).

But how to find the files?

Modern distributions (\texlive{} and \miktex{}, at least) provide the
means to update your system ``over the net''.  This is the minimum
effort route to getting a new file: `simply' find which of the
distribution's `packages' holds the file in question, and ask the
distribution to update itself.  The mechanisms are different (the two
distributions exhibit the signs of evolutionary divergence in their
different niches), but neither is difficult~--- see % ! line break
``\Qref*{using \miktex{} for installing}{Q-inst-miktex*}'' and % ! line break
``\Qref*{using \texlive{} for installing}{Q-inst-texlive}''.

There are packages, though, that aren't in the distribution you use
(or for which the distribution hasn't yet been updated to offer the
version you need).

Some sources, such as these \acro{FAQ} answers, provide links to
files: so if you've learnt about a package here, you should be able to
retrieve it without too much fuss.

Otherwise, \acro{CTAN} provides a full-text search, at its
\href{http://www.ctan.org/}{`central database'} site, as well as
topic- and author-based indexes and a link to browse the archive
itself.

% Two search mechanisms are offered: the simpler search, locating files
% by name, simply scans a list of files (\File{FILES.byname}~--- see
% below) and returns a list of matches, arranged neatly as a series of
% links to directories and to individual files.

% The more sophisticated search looks at the contents of each catalogue
% entry, and returns a list of catalogue entries that mention the
% keywords you ask for.

An alternative way to scan the catalogue is to use the catalogue's
% ! line break
\href{http://mirrors.ctan.org/help/Catalogue/bytopic.html}{``by topic'' index};
this is an older mechanism than the topic-based search (above), but is
well presented (even though its data has not been updated for some time).  

In fact, \ProgName{Google}, and other search engines, can be useful
tools.  Enter your search keywords, and you may pick up a package that
the author hasn't bothered to submit to \acro{CTAN}.

A user of \ProgName{Google} can restrict the search to
\acro{CTAN} by entering
\begin{quote}
  \texttt{site:ctan.org tex-archive \meta{search term(s)}}
\end{quote}
in \ProgName{Google}'s ``search box''.  You can also enforce the
restriction using \ProgName{Google}'s ``advanced search'' mechanism;
other search engines (presumably) have similar facilities.

Many people avoid the need to go over the network at all, for their
searches, by downloading the file list that the archives' web
file searches use.  This file, \File{FILES.byname},
presents a unified listing of the archive (omitting directory names and
cross-links).  Its companion \File{FILES.last07days} is also useful, to
keep an eye on the changes on the archive.  Since these files are
updated only once a day, a nightly automatic download (using
\ProgName{rsync}) makes good sense.
\begin{ctanrefs}
\item[FILES.byname]\CTANref{f-byname}
\item[FILES.last07days]\CTANref{f-last7}
\LastEdit{2014-03-19}
\end{ctanrefs}

\Question[Q-findfont]{Finding new fonts}

Nowadays, new fonts are seldom developed by industrious people using
\mf{}, but if such do appear, they will nowadays be distributed in
the same way as any other part of \alltex{} collections.  (An
historical review of Metafont fonts available is held on \acro{CTAN}
as ``\mf{} font list''.)

Nowadays, most new fonts that appear are only available in some
scalable outline form, and a large proportion is distributed under
commercial terms.  Such fonts will only make their way to the free
distributions (at least \texlive{} and \miktex{}) if their licensing
is such that the distributions can accept them.  Commercial fonts
(those you have to pay for) do not get to distributions, though
support for some of them is held by \acro{CTAN}.

Arranging for a new font to be usable by \alltex{} is very different,
depending on which type of font it is, and which \tex{}-alike engine
you are using; roughly speaking:
\begin{itemize}
\item MetaFont fonts will work without much fuss (provided their
  sources are in the correct place in the installation's tree);
  \tex{}-with-\ProgName{dvips}, and \pdftex{} are ``happy'' with them.
  While a new font will need `generating' (by running \mf{}, etc.),
  distributions are set up to do that ``on the fly'' and to save the
  results (for next time).
\item Adobe Type 1 fonts can be made to work, after \extension{tfm}
  and (usually) \extension{vf} files have been created from their
  metric (\extension{afm}) files; \extension{map} files also need to
  be created.  Such fonts will work with \pdftex{}, and with the
  (`vanilla')\alltex{} and \ProgName{dvips} combination.
\item TrueType fonts can be made to work with \pdftex{}~--- see
% ! line break
\href{http://www.radamir.com/tex/ttf-tex.htm}{Using TrueType fonts with \tex{}\dots{}}
  (a rather dated document, dicsussing use with \miktex{}~1.11).
\item TrueType and OpenType fonts are the usual sort used by \xetex{}
  and \luatex{}; while straightforward use is pretty easy, one is
  well-advised to use a package such as \Package{fontspec} to gain
  access to the full range of a font's capabilities.
\end{itemize}

The answer ``\Qref*{choice of scalable fonts}{Q-psfchoice}'' discusses
fonts that are configured for general (both textual and mathematical)
use with \AllTeX{}.  The list of such fonts is sufficiently short that
they \emph{can} all be discussed in one answer.
\begin{ctanrefs}
\item[fontspec.sty]\CTANref{fontspec}
\item[\nothtml{\rmfamily}\MF{} font list]\CTANref{mf-list}
\end{ctanrefs}
\LastEdit{2014-01-21}

\Question[Q-CD]{The \TeX{} collection}

If you don't have access to the Internet, there are obvious
attractions to \TeX{} collections on a disc.  Even those with net
access will find large quantities of \TeX{}-related files to hand a
great convenience.

The \TeX{} collection provides this, together with
ready-to-run \TeX{} systems for various architectures.  The collection
is distributed on \acro{DVD}, and contains:
\begin{itemize}
\item The \texlive{} distribution, including the programs themselves
  compiled for a variety of architectures; \texlive{} may be
  run from the \acro{DVD} or installed on hard disc.
\item Mac\TeX{}: an easy to install \TeX{} system for MacOS/X, based
  on \texlive{}; this distribution includes a native Mac installer,
  % ! line break
  \href{http://www.uoregon.edu/~koch/texshop/}{the \TeX{}Shop front-end}
  and other Mac-specific tools;
\item Pro\TeX{}t: an easy to install \TeX{} system for Windows, based
  on \miktex{}, based on an `active' document that guides
  installation; and
\item A snapshot of \acro{CTAN}.
\end{itemize}
A fair number of national \TeX{} User Groups, as well as \acro{TUG},
distribute copies to their members at no extra charge.  Some user
groups are also able to sell additional copies:
contact your local user group or \Qref*{\acro{TUG}}{Q-TUG*}.

You may also download disc images from \acro{CTAN}; the installation
disc, Pro\TeX{}t and Mac\TeX{} are all separately available.  Beware:
they are all pretty large downloads.  \texlive{}, once installed, may
be updated online.

More details of the collection are available from
\begin{hyperversion}
  \href{http://www.tug.org/texcollection/}{its own web page}
  on the \acro{TUG} site.
\end{hyperversion}
\begin{flatversion}
  its own web page on the
  \acro{TUG} site \URL{http://www.tug.org/texcollection/}
\end{flatversion}
\begin{ctanrefs}
\item[Mac\TeX{}]\CTANref{mactex}
\item[Pro\TeX{}t]\CTANref{protext}
\item[\texlive{} install image]\CTANref{texlive}
\end{ctanrefs}

