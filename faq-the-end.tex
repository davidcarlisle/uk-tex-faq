% $Id: faq-the-end.tex,v 1.3 2014/01/28 18:17:36 rf10 Exp rf10 $

\section{You're still stuck?}

\Question[Q-noans]{You don't understand the answer}

While the \acro{FAQ} maintainers don't offer a `help' service, they're
very keen that you understand the answers they've already written.
They're (almost) written ``in a vacuum'', to provide something to
cover a set of questions that have arisen; it's always possible that
they're written in a way that a novice won't understand them.

Which is where you can help the community.  Mail the 
\begin{wideversion}
  \href{mailto:faq-devel@tex.ac.uk}{maintainers}
\end{wideversion}
\begin{narrowversion}
  maintainers (at \Email{faq-devel@tex.ac.uk})
\end{narrowversion}
to report the answer that you find unclear, and (if you can) suggest
what we need to clarify.  Time permitting (the team is small and all
its members are busy), we'll try and clarify the answer.  This way,
with a bit of luck, we can together improve the value of this resource
to the whole community.

Note that the \acro{FAQ} development email address is not for
answering questions: it's for you to suggest which questions we should
work on, or new questions we should answer in future editions.

Those who simply ask questions at that address will be referred to
\mailto{texhax@tug.org} or to \Newsgroup{comp.text.tex}.

\Question[Q-newans]{Submitting new material for the \acro{FAQ}}

The \acro{FAQ} will never be complete, and we always expect that
there will be people out there who know better than we do about
something or other.  We always need to be put right about whatever
we've got wrong, and suggestions for improvements, particularly
covering areas we've missed, are always needed: mail anything you have
to the
\begin{wideversion}
  \href{mailto:faq-devel@tex.ac.uk}{maintainers}
\end{wideversion}
\begin{narrowversion}
  maintainers (at \Email{faq-devel@tex.ac.uk})
\end{narrowversion}

If you have actual material to submit, your contribution is more than
ever welcome.  Submission in plain text is entirely acceptable, but
if you're really willing, you may feel free to mark up your submission
in the form needed for the \acro{FAQ} itself.  The markup is a
strongly-constrained version of \LaTeX{}~--- the constraints come from
the need to translate the marked-up text to \acro{HTML} on the fly
(and hence pretty efficiently).  There is a file \File{markup-syntax}
in the \acro{FAQ} distribution that describes the structure of the
markup, but there's no real substitute for reading at least some of
the source (\File{faqbody.tex}) of the \acro{FAQ} itself.  If you
understand \ProgName{Perl}, you may also care to look at the
translation code in \File{texfaq2file} and \File{sanitize.pl} in the
distribution: this isn't the code actually used on the Web site, but
it's a close relation and is kept
up to date for development purposes.
\begin{ctanrefs}
\item[\nothtml{\rmfamily}\acro{FAQ} distribution]\CTANref{faq}
\end{ctanrefs}

\Question[Q-bug]{What to do if you find a bug}

For a start, make entirely sure you \emph{have} found a bug.
Double-check with books about \TeX{}, \LaTeX{}, or whatever you're using;
compare what you're seeing against the other answers above; ask every
possible person you know who has any \TeX{}-related expertise.
The reasons for all this caution are various.

If you've found a bug in \TeX{} itself, you're a rare animal indeed.
Don Knuth is so sure of the quality of his code that he offers real
money prizes to finders of bugs; the cheques he writes are
such rare items that they are seldom cashed. If \emph{you}
think you have found a genuine fault in \TeX{} itself (or \MF{}, or the
\acro{CM} fonts, or the \TeX{}book), don't immediately write to Knuth,
however. He only looks at bugs infrequently, and even then
only after they are agreed as bugs by a small vetting team. In the
first instance, contact Barbara Beeton at the \acro{AMS}
(\Email{bnb@ams.org}), or contact
\Qref*{\acro{TUG}}{Q-TUG*}.

If you've found a bug in \LaTeXe{}, \Qref*{report it}{Q-latexbug}
using mechanisms supplied by the \LaTeX{} team.  (Please be
careful to ensure you've got a \LaTeX{} bug, or a bug in one of the
``required''  packages distributed by the \LaTeX{} team.)

If you've found a bug in a contributed \LaTeX{} package, the best
starting place is usually to ask in the ``usual places for % ! line break
\Qref*{help on-line}{Q-gethelp}, or just possibly one of the % ! line break
\Qref*{specialised mailing lists}{Q-maillists*}.
The author of the package may well answer on-line, but if no-one
offers any help, you may stand a chance if you mail the author
(presuming that you can find an address\dots{}).

If you've found a bug in \LaTeXo{}, or some other such unsupported
software, your only real hope is \Qref*{help on-line}{Q-gethelp}.

Failing all else, you may need to pay for
help~--- \acro{TUG} maintains a % ! line break
\href{http://www.tug.org/consultants.html}{register of \TeX{} consultants}.
(This of course requires that you have the resources~--- and a
pressing  enough need~--- to hire someone.)


\htmlignore
\par
\endhtmlignore
\Question[Q-latexbug]{Reporting a \LaTeX{} bug}
\htmlignore
\par
\endhtmlignore
%
% This here isn't a reference to a question...
\label{lastquestion}

The \LaTeX{} team supports \LaTeX{}, and will deal with
% beware line wrap
\emph{bona fide} bug reports.  Note that the \LaTeX{} team does
\emph{not} deal with contributed packages~--- just the software that
is part of the \LaTeX{} distribution itself: \LaTeX{} and the
``required'' packages.
Furthermore, you need to be slightly
careful to produce a bug report that is usable by the team.  The steps
are:

\nothtml{\noindent}\textbf{1.} Are you still using current \LaTeX{}?  Maintenance is only
available for sufficiently up-to-date versions of \LaTeX{}~--- if your
\LaTeX{} is more than two versions out of date, the bug reporting
mechanisms may reject your report.

\nothtml{\noindent}\textbf{2.} Has your bug already been reported?  Browse the
% beware line break
\href{http://www.latex-project.org/cgi-bin/ltxbugs2html?introduction=yes}{\LaTeX{} bugs database},
to find any earlier instance of your bug.  In many cases, the database
will list a work-around.

\nothtml{\noindent}\textbf{3.} Prepare a % ! line break
\Qref*{``minimum'' file}{Q-minxampl} that exhibits the problem.
Ideally, such a file should contain no contributed packages~--- the
\LaTeX{} team as a whole takes no responsibility for such packages (if
they're supported at all, they're supported by their authors).  The
``minimum'' file should be self-sufficient: if a member of the team
runs it in a clean directory, on a system with no contributed
packages, it should replicate your problem.

\nothtml{\noindent}\textbf{4.} Run your file through \LaTeX{}: the bug
system needs the \extension{log} file that this process creates.

\nothtml{\noindent}\textbf{5.} Connect to the % ! line break
\href{http://www.latex-project.org/bugs-upload.html}{latex bugs processing web page}
and enter details of your bug~--- category, summary and full
description, and the two important files (source and log file).

The personal details are \emph{not} optional: the members of the
\LaTeX{} team may need to contact to discuss the bug with you, or to
advise you of a work-around.  Your details will not appear in the
public view of the database.

%%% Local Variables: 
%%% mode: latex
%%% TeX-master: t
%%% End: 
