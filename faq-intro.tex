% $Id: faq-intro.tex,v 1.31 2014/01/28 18:17:36 rf10 Exp rf10 $

% this environment is for the benefit of new-style file generation
\begin{introduction}
% suppress section numbers until end environment
\nothtml{\csname c@secnumdepth\endcsname=-1 } 
\section{Introduction}

This is a set of Frequently Asked Questions (\acro{FAQ}) for
English-speaking users of \TeX{}.

The questions answered here cover a wide range of topics, but
typesetting issues are mostly covered from the viewpoint of a \LaTeX{}
user.  Some of the questions answered have little relevance to today's
users; this is inevitable~--- it's easier to add information than it
is to decide that information is no longer needed.  The set of
answered questions is therefore in a state of slow flux: new questions
are answered, while old questions are deleted~\dots{}~but the whole
process depends on the time available for \acro{FAQ} maintenance.

\nothtml{\noindent} You may use the \acro{FAQ}
\begin{itemize}
\item by reading a printed document,
\item by viewing a \acro{PDF} file, with hyperlinks to assist
  browsing: copies are available formatted so that they could be
  printed on \CTANhref{faq-a4}{A4 paper} or on % ! line break
  \CTANhref{faq-letter}{North American ``letter'' paper}, \htmlonly{or}
\begin{typesetversion}
\item by using the \acro{FAQ}'s web interface (base \acro{URL}:
  \URL{http://www.tex.ac.uk/faq}); this version provides simple
  search capabilities, as well as a link to Google for a more
  sophisticated search restricted to the \acro{FAQ} itself, or
\end{typesetversion}
\item via Scott Pakin's \CTANhref{visualFAQ}{Visual FAQ}, which shows
  \LaTeX{} constructions with links to \acro{FAQ} explanations of how
  they may be created.
\end{itemize}

\subsection{Licence of the \acro{FAQ}}

The source of the \acro{FAQ}, available in the % ! line break
\CTANhref{faq}{\acro{FAQ}'s \ctan{} directory}, and its derived
representations (currently, the \acro{HTML} found at
\url{http://www.tex.ac.uk/faq} and \acro{PDF} copies, also in the
% ! line break
\CTANhref{faq}{\acro{FAQ}'s \ctan{} directory}) are all placed in the
public domain.

\subsection{Finding the Files}
\begin{typesetversion}
Unless otherwise specified, all files mentioned in this \acro{FAQ}
are available from a \ctan{} archive, or from a mirror of
\ctan{}~--- see \Qref[question]{later discussion}{Q-archives}% ! space
\narrowonly{, which gives details} of the \ctan{}
archives\wideonly{ and}\narrowonly{, including} how to retrieve files
from them.

The reader should also note that the first directory name of the path
name of every file on \ctan{} is omitted from what follows, for
the simple reason that, while it's always the same
(\path{tex-archive/}) on the main sites, mirror sites often choose
something else.

To avoid confusion, we also omit the full stop from the end of any
sentence whose last item is a path name (such sentences are rare, and
only occur at the end of paragraphs).  Though the path names are set
in a different font from running text, it's not easy to distinguish
the font of a single dot!
\end{typesetversion}
\begin{htmlversion}
Almost all files suggested in any answer in these \acro{FAQ}s are
available from \Qref{\ctan{} archives}{Q-archives}, and may be
reached via links in the file list at the end of the answer.  In
particular, documentation (when available) is linked from the
\ctan{} Catalogue entry, which is also listed in the file list.

Unless doing so is unavoidable, the answers do not mention things that
are not on the \ctan{} archives.  An obvious exception is web
resources such as supplementary documents, etc.
\end{htmlversion}

\begin{typesetversion}
  \subsection*{Origins}
\end{typesetversion}
\begin{htmlversion}
  \textbf{Origins}
\end{htmlversion}
The \acro{FAQ} was originated by the Committee of the \acro{UK}
\TeX{} Users' Group (\acro{UK}~\acro{TUG}), in 1994, as a development of a
regular posting to the \emph{Usenet} newsgroup
\Newsgroup{comp.text.tex} that was maintained for some time by Bobby
Bodenheimer.  The first \acro{UK} version was much re-arranged and
corrected from the original, and little of Bodenheimer's work now
remains.

\htmlignore
The following people (at least~--- there are almost certainly others
whose names weren't correctly recorded) have contributed help or
advice on the development of the \acro{FAQ}:
William Adams, % spotting dead url
Donald Arseneau, % indefatigable
Rosemary Bailey, % lots of initial set of maths answers
Barbara Beeton, % one of first reviewers outside cttee
Karl Berry, % often provides comments & suggestions
Giuseppe Bilotta,
Charles Cameron,
Fran\c{c}ois Charette, % update of q-xetex
Damian Cugley,
Martin Garrod, % two consecutive "report" in clsvpkg
Michael Dewey,
Michael Downes, % (RIP) on (ams)maths stuff
Jean-Pierre Drucbert, % (RIP) corrected my interpretation of minitoc
David Epstein, % comment on q-algorithms, suggestion of tex-sx
Michael Ernst, % alert on confusing colloquialism
Thomas Esser,
Ulrike Fischer, % spotted boondoggle in q-tabcellalign
                % args like \section improvement; general typos
Bruno Le Floch, % alert to mess in q-latex3
Anthony Goreham,
Norman Gray,
Enrico Gregorio, % new definition for q-seccntfmt
Werner Grundlingh, % xparse for *-commands
Eitan Gurari, % (RIP) comparative html translators
William Hammond, % lots of work on xml-related answers
John Hammond, % corrections to q-protect
John Harper, % patient help with q-robust ;-)
Gernot Hassenpflug, % impetus and suggestions for q-printvar
Troy Henderson, % metapost tutorials update
Hartmut Henkel,
Stephan Hennig, % text about vc in Q-RCS
John Hobby,
Morten H\textoslash gholm, % eagle eyes in search for problems
Berthold Horn, % lots of stuff on fonts
Ian Hutchinson, % comparative html translators
Werner Icking, % (RIP) answer on tex-music
William Ingram, % extension of transcinit
David Jansen, % error in errmissitem example code
Alan Jeffrey, % another founding father
Regnor Jernsletten,
David Kastrup, % inter alia, suggested mention of lilypond
Oleg Katsitadze, % revision of q-eplain
Isaac Khabaza, % kindersley book
Ulrich Klauer, % pointed out twaddle in TeXpronounce
Markus Kohm, % help with functionality of Koma-script
Stefan Kottwitz % picked up text of q-wideflt
David Kraus, % help with tocbibind (spotted embarrassing typo :-} )
Ryszard Kubiak, % comment about q-driver
Simon Law,
Uwe L\"uck, % correction of typo 
Daniel Luecking, % typos
Aditya Mahajan, % rewrite of q-context
Sanjoy Mahajan,
Diego Andres Alvarez Marin, % suggestion of lacheck
Andreas Matthias, % most recently q-isitanum
Steve Mayer, % proof reading
Javier Mora, % bug in q-cmdstar
Brooks Moses, % editorial suggestion
Peter Moulder, % improvement of q-widows
Iain Murray, % alternative solution in q-hyperdupdest
Vilar Camara Neto, % explain breakurl pkg, suggest use of numprint
Dick Nickalls,
Ted Nieland,
Hans Nordhaug,
Pat Rau,
Heiko Oberdiek,
Piet van Oostrum,
Scott Pakin, % bug reports (at least)
Oren Patashnik,
Manuel P\'egouri\'e-Gonnard, % improvement of q-luatex
Steve Peter, % comprehensive review, lots of typos etc
Sebastian Rahtz, % without whom there would be no...
Philip Ratcliffe, % help with a bibtex oddity, editorial comment
Chris Rowley, % founding father
Jos\'e Carlos Santos, % assiduous reader of new versions
Walter Schmidt,
Hans-Peter Schr\"ocker,
Joachim Schrod,
Uwe Siart, % correction for manymathalph
Maarten Sneep, % summary of c.t.t thread for outszwrng, inter alia
Axel Sommerfeldt, % suggested rewrite of q-hyperdupdest
Philipp Stephani, % xlation of komascript keyval stuff
James Szinger, % who responded on c.t.t to request for answer
Nicola Talbot, % correction of link
Bob Tennent, % urw classico + sansmath
Tomek Trzeciak, % suggested docmute and standalone together
Ulrik Vieth,
Mike Vulis, % details of micropress offerings
Chris Walker, % revtex alert, ite suggestion (q-labelfig)
Peter Wilson, % several, incl functionality of memoir
Joseph Wright, % help with keyval, l3keys, etc.
Rick Zaccone,
Gregor Zattler % typo spotted
and
Reinhard Zierke.

That list does \emph{not} cover the many people whose ideas were
encountered on various mailing lists, in newsgroups, or (more
recently!) in web forums.  Many 
such people have helped, even if simply to highlight an area in which
\acro{FAQ} work would be useful.
\endhtmlignore
% \LastEdit{2013-08-20} % not to be typeset or converted to html...
\end{introduction}
